\documentclass[t, 10pt, mathserif]{beamer}

\usepackage[spanish]{babel}
\usepackage[utf8]{inputenc}
\usepackage{amscd}
\usepackage{amssymb}
\usepackage{amsthm}
\usepackage{latexsym}
\usepackage{setspace}
\usepackage{url}
\usepackage{xcolor}
\usepackage{bbm}

\mode<presentation>{
	\usecolortheme{}
	\useinnertheme{}
	\useoutertheme{}
}

\beamertemplatenavigationsymbolsempty
\setbeamersize{text margin left = 1cm}
\setbeamertemplate{caption}[numbered]
\setbeamertemplate{frametitle}[default][left,leftskip=0.5cm]
\addtobeamertemplate{frametitle}{\vspace*{0.5cm}}{\vspace*{1cm}}

\setlength{\parskip}{\baselineskip}
\expandafter\def\expandafter\item\expandafter{\item \setlength{\parskip}{0.5\baselineskip}}
\expandafter\def\expandafter\definition\expandafter{\definition \setlength{\parskip}{0.5\baselineskip}}

\languagepath{spanish}
\deftranslation[to=spanish]{Corollary}{Corolario}
\deftranslation[to=spanish]{corollary}{corolario}
\deftranslation[to=spanish]{Definition}{Definición}
\deftranslation[to=spanish]{definition}{definición}
\deftranslation[to=spanish]{Lemma}{Lema}
\deftranslation[to=spanish]{lemma}{lema}
\deftranslation[to=spanish]{Problem}{Problema}
\deftranslation[to=spanish]{problem}{problema}
\deftranslation[to=spanish]{Theorem}{Teorema}
\deftranslation[to=spanish]{theorem}{teorema}
\newtheorem{belief}{Belief}
\newtheorem{conjecture}{Conjecture}
\newtheorem{observation}{Observation}
\newtheorem{question}{Question}

\newcommand {\base}[2]{\langle{#1};{#2}\rangle}
\newcommand{\abs}[1]{\left| #1 \right|}
\newcommand{\alocc}[2]{|\!|#1|\!|_{#2}}
\newcommand{\card}{\mbox{\raisebox{.13em}{{$\scriptstyle \#$}}}}
\newcommand{\ceil}[1]{\lceil #1 \rceil }
\newcommand{\cf}{\text{\em cf}}
\newcommand{\eps}{\varepsilon}
\newcommand{\expa}[1]{\{#1\}}
\newcommand{\floor}[1]{\lfloor #1 \rfloor } 
\newcommand{\N}{{\mathbb{N}}}
\newcommand{\NN}{\mathbb{N}}
\newcommand{\occ}[2]{|#1|_{#2}}
\newcommand{\Q}{{\mathbb{Q}}}
\newcommand{\R}{{\mathbb{R}}}
\newcommand{\RR}{\mathbb{R}}
\newcommand{\uno}{\mathbbm{1}}
\newcommand{\wh}[1]{\widehat{#1}}
\newcommand{\xbar}{{\overline{x}}}
\newcommand{\ybar}{{\overline{y}}}
\newcommand{\Z}{{\mathbb{Z}}}

\author{\large Emilio Almansi}

\begin{document}

\title{\normalsize Tesis de Licenciatura en Ciencias de la Computación \vspace*{1.5cm}}

\subtitle{\Large Secuencias completamente equidistribuidas basadas en secuencias de De Bruijn}

\date{
  {\footnotesize
    \hspace*{-6cm}
    \begin{tabular}{l}
      Directora: Verónica Becher \\
      Departamento de Computación\\
      Facultad de Ciencias Exactas y Naturales\\
      Universidad de Buenos Aires\\
      4 de septiembre, 2019
    \end{tabular}
  }
}

%%%%%%%%%%%%%%%%%%%%%%%%%%%%%%%%%%%%%%%%%%%%%%%%%%%%%%%%%%%%%%%%%%%

\begin{frame}
  \vspace{0.5cm}
  \maketitle
  \setcounter{framenumber}{0}
  \thispagestyle{empty}
\end{frame}

%%%%%%%%%%%%%%%%%%%%%%%%%%%%%%%%%%%%%%%%%%%%%%%%%%%%%%%%%%%%%%%%%%%

\begin{frame}
  \frametitle{Sobre secuencias aleatorias {$^{(1)}$}}

  ¿Qué tienen en común las siguientes disciplinas?
  \pause

  \begin{itemize}
    \item Criptografía, seguridad informática.\pause
    \item Predicción del clima, medicina nuclear, simulación de proteínas.\pause
    \item Aprendizaje automático, algoritmos probabilistas.\pause
    \item Juegos de azar, videojuegos, simulaciones físicas.
  \end{itemize}
  \pause

  {\color{orange} Generación de números aleatorios.}
  \pause

  Pero, ¿qué es una secuencia de números aleatorios?
\end{frame}

%%%%%%%%%%%%%%%%%%%%%%%%%%%%%%%%%%%%%%%%%%%%%%%%%%%%%%%%%%%%%%%%%%%

\begin{frame}
  \frametitle{Sobre secuencias aleatorias {$^{(2)}$}}

  \textbf{Intuición}: si tiro un dado muchas veces seguidas, el resultado de cada tirada tiene que ser \textit{impredecible} y todo número del 1 al 6 tiene que ser \textit{equiprobable}.
  \pause

  Respecto a la parte de \textit{impredecible}:
  \begin{quote}
    If ``random`` means that the sequence satisfies no predictable rules, the title of this paper is contradictory.
    \begin{flushright}
      \small{Construction of a Random Sequence \\}
      \small{Donald Knuth, 1965}
    \end{flushright}
  \end{quote}
  \pause

  En este trabajo, nos enfocamos en la parte de \textit{equiprobable}.
  % Es posible construir secuencias determinísticas que cumplen con esta propiedad.
\end{frame}

%%%%%%%%%%%%%%%%%%%%%%%%%%%%%%%%%%%%%%%%%%%%%%%%%%%%%%%%%%%%%%%%%%%

\begin{frame}
  \frametitle{Equidistribución {$^{(1)}$}}

  \begin{definition}
    Dado un entero $b$, una secuencia de \textit{números enteros} $X = x_1, x_2, \dots$ del conjunto $\{ 0, 1, \dots, b - 1 \}$ es \textbf{equidistribuida} si todo valor posible aparece con frecuencia asintótica igual a $\frac{1}{b}$:
    \pause

    \begin{equation*}
      \begin{aligned}
        Pr \left( x_i = j \right) = \frac{1}{b} \hspace*{1cm} \text{para todo } j \in \{ 0, \dots, b - 1 \} \text{,}
      \end{aligned}
    \end{equation*}
    \pause

    donde $Pr(x_i = j) = \lim_{N \to \infty} \frac{1}{N} \sum_{i = 1}^{N} \sigma(x_i = j)$.
  \end{definition}
  \pause

  Ahora, definimos una noción equivalente para secuencias de números reales.
\end{frame}

%%%%%%%%%%%%%%%%%%%%%%%%%%%%%%%%%%%%%%%%%%%%%%%%%%%%%%%%%%%%%%%%%%%

\begin{frame}
  \frametitle{Equidistribución {$^{(2)}$}}

  \begin{definition}
    Una secuencia de \textit{números reales} $X = x_1, x_2, \dots$ en el intervalo unitario $\left[0, 1\right)$ es \textbf{equidistribuida} si, dado cualquier conjunto $I \subseteq \left[0, 1\right)$, la frecuencia asintótica con la que la secuencia toma valores en $I$ es igual a su tamaño:
    \pause

    \begin{equation*}
      \begin{aligned}
        Pr \left( x_i \in I\, \right) = \left| I\, \right| \hspace*{1cm} \text{para todo } I \subseteq \left[0, 1\right) \text{,}
      \end{aligned}
    \end{equation*}
    \pause

    donde $Pr \left( x_i \in I\, \right) = \lim_{N \to \infty} \frac{1}{N} \sum_{i = 1}^{N} \sigma(x_i \in I\,)$.
  \end{definition}
  \pause

  Si $X$ es equidistribuida en $[0, 1)$, entonces para cualquier $b$ la secuencia $Y = \left( \lfloor b x_i \rfloor \right)_{i = 1}^{\infty}$ es equidistribuida en $\{ 0, 1, \dots, b - 1 \}$.
\end{frame}

%%%%%%%%%%%%%%%%%%%%%%%%%%%%%%%%%%%%%%%%%%%%%%%%%%%%%%%%%%%%%%%%%%%

\begin{frame}
  \frametitle{$k$-distribución {$^{(1)}$}}

  % Ahora extendemos la noción de equidistribución a otras dimensiones.
  % \pause

  \textbf{Intuición}: cualquier seguidilla de tiradas tiene que aparecer con igual frecuencia. Por ejemplo, (1, 1) aparece con la misma frecuencia que (2, 2) y que (6, 4).
  \pause

  Trabajamos con ``ventanas`` de tamaño $k$ de la secuencia. Si $X = x_1, x_2, \dots$ es una secuencia de \textit{números reales}, entonces:
  \pause

  \vspace{-0.3cm}
  \begin{equation*}
    \begin{aligned}
        w_1 & = (x_1, x_2, \dots, x_k      ), \\
        w_2 & = (x_2, x_3, \dots, x_{k + 1}), \\
        w_3 & = (x_3, x_4, \dots, x_{k + 2}), \\
            & \dots
    \end{aligned}
  \end{equation*}

  es la secuencia de ventanas de $X$, que llamamos $W_k(X)$.
\end{frame}

%%%%%%%%%%%%%%%%%%%%%%%%%%%%%%%%%%%%%%%%%%%%%%%%%%%%%%%%%%%%%%%%%%%

\begin{frame}
  \frametitle{$k$-distribución {$^{(2)}$}}

  \begin{definition}
    Una secuencia de \textit{números reales} $X = x_1, x_2, \dots$ en el intervalo unitario $\left[0, 1\right)$ es \textbf{$k$-distribuida} si, dado cualquier conjunto $I \subseteq \left[0, 1\right)^k$, la frecuencia asintótica con la que la secuencia de ventanas de $X$ toma valores en $I$ es igual a su tamaño:
    \pause

    \begin{equation*}
      \begin{aligned}
        Pr \left( w_i \in I\, \right) = \left| I\, \right| \hspace*{1cm} \text{para todo } I \subseteq \left[0, 1\right)^k \text{,}
      \end{aligned}
    \end{equation*}
    \pause

    \vspace{-0.5cm}
    \begin{equation*}
      \hspace{-2.5cm}
      \begin{aligned}
          \text{donde } W_k(X) =\; & \left( w_i \right)_{i = 1}^{\infty} \\
                               =\; & (x_1, x_2, \dots, x_k      ),\, (x_2, x_3, \dots, x_{k + 1}),\, \dots \, \text{.}
      \end{aligned}
    \end{equation*}
    \vspace{-0.3cm}
  \end{definition}
  \pause

  Si $X$ es $k$-distribuida para todo $k$, entonces $X$ es \textbf{completamente equidistribuida}.
\end{frame}

% %%%%%%%%%%%%%%%%%%%%%%%%%%%%%%%%%%%%%%%%%%%%%%%%%%%%%%%%%%%%%%%%%%%

\begin{frame}
  \frametitle{Equidistribución completa}

  ¿Por qué estudiar la propiedad de equidistribución?

  \begin{itemize}
    \item Requerimiento básico de pseudo-aleatoriedad. Propiedades de equipartición y autocorrelación con retraso. \pause
    \item Calidad de un generador de números aleatorios o PNRG. Pruebas de aleatoriedad.\pause % Mersenne Twister
    \item Integración de Montecarlo, criterio de la integral de Riemann. \pause
    \item Vínculo: teoría de números, computación, probabilidad y estadística.
  \end{itemize}
  \pause

  ¿Qué \textbf{no} es la equidistribución?
\end{frame}
 
% %%%%%%%%%%%%%%%%%%%%%%%%%%%%%%%%%%%%%%%%%%%%%%%%%%%%%%%%%%%%%%%%%%%

\begin{frame}
  \frametitle{Discrepancia}

  \vspace{-0.5cm}
  \begin{figure}
    \includegraphics[width=\textwidth]{resources/discrepancia.png}
    \caption{Izq.) Secuencia Sobol 2,3. Der.) Secuencia pseudo-aleat.}
  \end{figure}
\end{frame}
 
%%%%%%%%%%%%%%%%%%%%%%%%%%%%%%%%%%%%%%%%%%%%%%%%%%%%%%%%%%%%%%%%%%%

\begin{frame}{Secuencias de De Bruijn}
  \only<1->{
    Son secuencias muy estudiadas en combinatoria. Tienen la propiedad de ``contener`` a todas las posibles combinaciones de secuencias de un alfabeto y tamaño dados. Ejemplos:
  }

  \only<2->{Con $b = 2, k = 3$:}
  \only<2->{\vspace{-0.3cm}}
  \begin{center}
    \only<2>{$0, 0, 0, 1, 0, 1, 1, 1$}
    \only<3>{$\textbf{0, 0, 0}, 1, 0, 1, 1, 1$}
    \only<4>{$0, \textbf{0, 0, 1}, 0, 1, 1, 1$}
    \only<5>{$0, 0, \textbf{0, 1, 0}, 1, 1, 1$}
    \only<6>{$0, 0, 0, \textbf{1, 0, 1}, 1, 1$}
    \only<7>{$0, 0, 0, 1, \textbf{0, 1, 1}, 1$}
    \only<8>{$0, 0, 0, 1, 0, \textbf{1, 1, 1}$}
    \only<9>{$\textbf{0}, 0, 0, 1, 0, 1, \textbf{1, 1}$}
    \only<10>{$\textbf{0, 0}, 0, 1, 0, 1, 1, \textbf{1}$}
    \only<11->{$0, 0, 0, 1, 0, 1, 1, 1$}
    \only<11->{\vspace{-0.3cm}}
  \end{center}

  \only<11->{Con $b = 4, k = 2$:}
  \only<11->{\vspace{-0.3cm}}
  \begin{center}
    \only<12->{$0, 0, 1, 0, 2, 0, 3, 1, 1, 2, 1, 3, 2, 2, 3, 3$}
  \end{center}

  \only<13->{Notar que siempre tienen longitud $b^k$.}
  \only<14->{{\color{orange} ¿Qué relación tienen con las secuencias equidistribuidas?}}
\end{frame}

%%%%%%%%%%%%%%%%%%%%%%%%%%%%%%%%%%%%%%%%%%%%%%%%%%%%%%%%%%%%%%%%%%%

\begin{frame}
  \frametitle{Secuencia de Knuth {$^{(1)}$}}

  \vspace{-0.4cm}
  \begin{definition}
    Una \textbf{secuencia $A$ de órden $n$} es la secuencia que se obtiene al dividir cada elemento de una secuencia de De Bruijn de base $2^n$ y de órden $n$ por su base:
    \pause

    \vspace{-0.2cm}
    \begin{equation*}
      \begin{aligned}
        A^{(n)} & = \frac{f_1}{2^n}, \frac{f_2}{2^n}, \dots, \frac{f_{2^{n^2}}}{2^n}
      \end{aligned}
    \end{equation*}
    \pause

    \vspace{-0.2cm}
    donde $F^{(2^n, n)} = f_1, \dots, f_{2^{n^2}}$ denota una secuencia de De Bruijn de base $2^n$ y de órden $n$.
    \pause

    Una \textbf{secuencia $B$ de órden $n$} es la secuencia que se obtiene de concatenar $n 2^{2 n}$ copias de una secuencia $A$ de órden $n$:
    \pause

    \vspace{-0.2cm}
    \begin{equation*}
      B^{(n)} = \left< \underbrace{A^{(n)} ; A^{(n)} ; \dots ; A^{(n)}}_{n 2^{2 n} \text{ veces}} \right> \text{.}
    \end{equation*}
  \end{definition}
\end{frame}


%%%%%%%%%%%%%%%%%%%%%%%%%%%%%%%%%%%%%%%%%%%%%%%%%%%%%%%%%%%%%%%%%%%

\begin{frame}
  \frametitle{Secuencia de Knuth {$^{(2)}$}}

  Ejemplo para $n = 2$:
  \pause

  \begin{equation*}
    \begin{aligned}
      F^{(4, 2)} & = 0, 0, 1, 0, 2, 0, 3, 1, 1, 2, 1, 3, 2, 2, 3, 3 \\[0.4cm] \pause
      A^{(2)}    & = \frac{0}{4}, \frac{0}{4}, \frac{1}{4}, \frac{0}{4}, \frac{2}{4}, \frac{0}{4}, \frac{3}{4}, \frac{1}{4}, \frac{1}{4}, \frac{2}{4}, \frac{1}{4}, \frac{3}{4}, \frac{2}{4}, \frac{2}{4}, \frac{3}{4}, \frac{3}{4} \\[0.4cm] \pause
      B^{(2)}   & = \left< \underbrace{A^{(2)} ; \dots ; A^{(2)}}_{2 \times 2^{2 \times 2} = 32 \text{ veces}} \right> \\[0.1cm] \pause
                & = \underbrace{\frac{0}{4}, \frac{0}{4}, \dots, \frac{3}{4}, \frac{3}{4}}_{A^{(2)}}, \dots, \underbrace{\frac{0}{4}, \frac{0}{4}, \dots, \frac{3}{4}, \frac{3}{4}}_{A^{(2)}} \text{.}
    \end{aligned}
  \end{equation*}
\end{frame}

%%%%%%%%%%%%%%%%%%%%%%%%%%%%%%%%%%%%%%%%%%%%%%%%%%%%%%%%%%%%%%%%%%%

\begin{frame}
  \frametitle{Secuencia de Knuth {$^{(3)}$}}

  Ahora sí, ya estamos en condiciones de definir la secuencia de Knuth.
  \pause

  \medskip
  \begin{definition}
    La secuencia de Knuth, que denominamos $K$, se define como la concatenación de todas las posibles secuencias $B$ en órden creciente:
    \pause

    \begin{equation*}
      K = \left< B^{(1)} ; B^{(2)} ;  B^{(3)} ; \dots \right> \text{.}
    \end{equation*}
  \end{definition}
  \pause

  \begin{theorem}[{{\scriptsize  Knuth, 1965}}]
    La secuencia $K$ es completamente equidistribuida.
  \end{theorem}
\end{frame}

%%%%%%%%%%%%%%%%%%%%%%%%%%%%%%%%%%%%%%%%%%%%%%%%%%%%%%%%%%%%%%%%%%%

\begin{frame}
  \frametitle{Idea de la demostración}

  Knuth primero formula el lema 2 y reduce el problema a probar la $k$-distribución de una familia de secuencias de números enteros.
  \pause
  
  Para cada entero $m$, prueba que la secuencia $\lfloor 2^m K \rfloor$ de números enteros es $k$-distribuida para todo $k$. Esto implica que $K$ también lo es.
  \pause

  \medskip
  \begin{definition}
    Fusce sit amet lacus viverra, viverra massa sit amet, placerat neque. Integer ipsum sapien, efficitur quis dui vitae, facilisis tempus dolor.
    \pause

    Duis ornare volutpat libero, at sodales dolor porttitor at.
    \pause
  \end{definition}

  In rutrum dapibus justo, at mattis lacus ultrices sed. Suspendisse suscipit luctus fermentum.
\end{frame}


% %%%%%%%%%%%%%%%%%%%%%%%%%%%%%%%%%%%%%%%%%%%%%%%%%%%%%%%%%%%%%%%%%%%

% \begin{frame}
%   \frametitle{Secuencias de De Bruijn}

%   \medskip
%   \begin{definition}[{{\scriptsize  Borel, 1909}}]
%     Mauris euismod neque a lorem rutrum, id molestie eros consequat. In facilisis magna eu libero commodo, id tincidunt {\color{magenta} $\ell$} purus pellentesque:
%     \begin{equation*}
%       \lim_{n \rightarrow \infty} \frac{\alocc{u[1,\ell n]}{v}}{n} = \frac{1}{|A|^{\ell}}.  
%     \end{equation*}
%     \pause

%     Donec nec ex id nisl venenatis semper. Curabitur erat mi, sagittis nec tortor vel, tempor porta magna. Cras at maximus orci, non viverra neque.
%   \end{definition}
%   \pause

%   \medskip
%   \begin{problem}[{{\scriptsize  Borel, 1909}}]
%     In eget enim feugiat, cursus tellus eget, dapibus libero. Class aptent taciti sociosqu ad litora torquent per conubia nostra, per inceptos himenaeos.
%   \end{problem}
% \end{frame}

% %%%%%%%%%%%%%%%%%%%%%%%%%%%%%%%%%%%%%%%%%%%%%%%%%%%%%%%%%%%%%%%%%%%

% \begin{frame}
%   \frametitle{La secuencia de Knuth {$^{(1)}$}}

%   Nam sagittis dolor in enim tincidunt, sit amet pellentesque urna.

%   \begin{equation*}
%     \begin{aligned}
%       & F^{(2^n, n)}  & = \; & f_1, \dots, f_{2^{n^2}} \\[0.3cm]
%       \pause
%       & A^{(n)}       & = \; & \frac{f_1}{2^n}, \frac{f_2}{2^n}, \dots, \frac{f_{2^{n^2}}}{2^n} \\
%       &               & = \; & \bigg( \frac{f_i}{2^n} \bigg)_{i = 1}^{2^{n^2}} \\[0.3cm]
%       \pause
%       & B^{(n)}       & = \; & \left< \underbrace{A^{(n)} ; A^{(n)} ; \dots ; A^{(n)}}_{n 2^{2 n} \text{ veces}} \right>
%     \end{aligned}
%   \end{equation*}
%   \pause

%   Porta purus neque, ultrices vulputate orci ullamcorper eu.
% \end{frame}

%   % \medskip
%   % \begin{problem}
%   %   Morbi euismod purus at cursus iaculis. Donec efficitur lorem rutrum, auctor justo id, rhoncus nibh.
%   % \end{problem}
%   % \pause

%   % \medskip
%   % \begin{theorem}[{{\scriptsize  Champernowne, 1933}}]
%   %   Curabitur varius in ligula nec laoreet. Aenean ultricies eget mi quis maximus. Mauris ornare interdum vestibulum.
%   % \end{theorem}

% %%%%%%%%%%%%%%%%%%%%%%%%%%%%%%%%%%%%%%%%%%%%%%%%%%%%%%%%%%%%%%%%%%%

% \begin{frame}
%   \frametitle{La secuencia de Knuth {$^{(2)}$}}

%   For example, when $n = 2$:

%   \begin{equation*}
%     \begin{aligned}
%       & F^{(4, 2)} = 0, 0, 1, 0, 2, 0, 3, 1, 1, 2, 1, 3, 2, 2, 3, 3 \\
%       \pause
%       & F^{(4, 2)} = 0, 0, 1, 0, 2, 0, 3, 1, 1, 2, 1, 3, 2, 2, 3, 3 \\
%     \end{aligned}
%   \end{equation*}
%   % \begin{equation*}
%   %   \begin{aligned}
  

%   %     & B^{(2)} = \left< \underbrace{A^{(2)} ; \dots ; A^{(2)}}_{32 \text{ times}} \right> = \underbrace{\frac{0}{4}, \frac{0}{4}, \dots, \frac{3}{4}, \frac{3}{4}}_{A^{(2)}}, \dots, \underbrace{\frac{0}{4}, \frac{0}{4}, \dots, \frac{3}{4}, \frac{3}{4}}_{A^{(2)}}
%   %   \end{aligned}
%   % \end{equation*}
%   % \pause

%   and $|A^{(2)}| = 16$, $|B^{(2)}| = 512$.
% \end{frame}

% %%%%%%%%%%%%%%%%%%%%%%%%%%%%%%%%%%%%%%%%%%%%%%%%%%%%%%%%%%%%%%%%%%%

% \begin{frame}
%   \frametitle{La secuencia de Knuth (3)}

%   Curabitur varius in ligula nec laoreet.
%   \begin{equation*}
%     K = \left< B^{{$^{(1)}$}} ; B^{(2)} ;  B^{(3)} ; \dots \right>
%   \end{equation*}
%   \pause

%   \medskip
%   \begin{theorem}[{{\scriptsize  Knuth, 1965}}]
%     Nam sagittis dolor in enim tincidunt, sit amet pellentesque urna vulputate.
%   \end{theorem}
%   \pause
  
%   Morbi euismod purus at cursus iaculis. Donec efficitur lorem rutrum, auctor justo id, rhoncus nibh.
%   \pause

%   Aenean ultricies eget mi quis maximus. Mauris ornare interdum vestibulum.
% \end{frame}

%   % \medskip
%   % \begin{problem}
%   %   Morbi euismod purus at cursus iaculis. Donec efficitur lorem rutrum, auctor justo id, rhoncus nibh.
%   % \end{problem}
%   % \pause

% %%%%%%%%%%%%%%%%%%%%%%%%%%%%%%%%%%%%%%%%%%%%%%%%%%%%%%%%%%%%%%%%%%%

% \begin{frame}
%   \frametitle{Tamaños de alfabeto linealmente crecientes}

%   \medskip
%   \begin{theorem}[{{\scriptsize  Agafonov 1968}}]
%     Class aptent taciti sociosqu ad litora torquent per conubia nostra, per inceptos himenaeos
%   \end{theorem}
%   \pause

%   \medskip
%   \begin{corollary}
%     Vestibulum quis dolor quam. Sed viverra, diam ac fringilla fringilla, ex dui consequat leo, nec tempus augue mi eu quam.
%   \end{corollary}
%   \pause

%   \medskip
%   \begin{theorem}[{{\scriptsize  Vandehey 2016}}]
%     Phasellus quis aliquam nulla, non rutrum lorem. Class aptent taciti sociosqu ad litora torquent per conubia nostra, per inceptos himenaeos.
%   \end{theorem}
% \end{frame}

% %%%%%%%%%%%%%%%%%%%%%%%%%%%%%%%%%%%%%%%%%%%%%%%%%%%%%%%%%%%%%%%%%%%

% \begin{frame}
%   \frametitle{Teorema principal}

%   Nullam posuere tincidunt urna et elementum. Donec elementum at tellus sit amet tempus.
%   \pause

%   \medskip
%   \begin{problem}
%     Cras id accumsan risus, sed elementum elit. Suspendisse aliquet hendrerit gravida.
%   \end{problem}
%   \pause

%   \medskip
%   \begin{theorem}[\color{magenta} Resultado principal de esta tesis\color{black}]
%     Nullam vehicula erat ante, hendrerit euismod elit luctus nec. Duis sagittis tincidunt metus, in dapibus lorem ullamcorper ut.
%   \end{theorem}
% \end{frame}

% %%%%%%%%%%%%%%%%%%%%%%%%%%%%%%%%%%%%%%%%%%%%%%%%%%%%%%%%%%%%%%%%%%%

% \begin{frame}
%   \frametitle{Idea de la demostración}

%   \medskip
%   \begin{definition}
%     Curabitur imperdiet tempus massa $A=\{0,1, \ldots, b-1\}$, pellentesque id turpis at mauris tempor auctor at pellentesque ex.
%     \pause

%     Pellentesque urna arcu, pellentesque sit amet volutpat eget, venenatis sed leo. Phasellus tempus eu urna a lacinia.
%     \pause

%     Vestibulum aliquam augue et tortor pulvinar suscipit $w^{\star}_n$.
%     \pause

%     Lorem ipsum dolor sit amet, consectetur adipiscing elit. Sed placerat nulla a vulputate ultrices. Ut et magna ac lacus elementum tincidunt a id ante.

%     {\color{magenta}
%       Donec $u=v_1 v_2 \ldots v_m$ aenean ullamcorper odio vitae $v_i$ erat $\ell_n$  rhoncus quis
%       \[
%       e_n(u)=e_n(v_1)\ldots e_n(v_m)
%       \]
%     }
%   \end{definition}
% \end{frame}

% %%%%%%%%%%%%%%%%%%%%%%%%%%%%%%%%%%%%%%%%%%%%%%%%%%%%%%%%%%%%%%%%%%%

% \begin{frame}
%   \frametitle{Criterio de Weyl}

%   \medskip
%   \begin{definition}
%     Suspendisse ut hendrerit $A$,  \textit{finibus semper neque non congue tortor dictum $\ell$} laoreet ex nec pulvinar $u\in A^*$ tellus
%     \begin{equation*}
%       \Delta_{A,\ell}(u) = 
%       \max_{v \in A^{\ell}}\left(\left|\frac{\alocc{u}{v}}{\lfloor|u|/\ell \rfloor} - \frac{1}{|A|^{\ell}}\right|\right).
%     \end{equation*}
%   \end{definition}
%   \pause

%   Nam sagittis dolor in enim tincidunt $v\in A^\omega$ sit amet pellentesque urna vulputate $\ell$
%   $$\lim_{n \rightarrow \infty} \Delta_{A, \ell}(v[1,\ell n]) = 0$$
% \end{frame}


% %%%%%%%%%%%%%%%%%%%%%%%%%%%%%%%%%%%%%%%%%%%%%%%%%%%%%%%%%%%%%%%%%%%

% \begin{frame}
%   \frametitle{Prueba alternativa}

%   \medskip
%   \begin{definition}
%     Curabitur imperdiet tempus massa $A=\{0,1, \ldots, b-1\}$, pellentesque id turpis at mauris tempor auctor at pellentesque ex.
%     \pause

%     Pellentesque urna arcu, pellentesque sit amet volutpat eget, venenatis sed leo. Phasellus tempus eu urna a lacinia.
%     \pause

%     Vestibulum aliquam augue et tortor pulvinar suscipit $w^{\star}_n$.
%     \pause

%     Lorem ipsum dolor sit amet, consectetur adipiscing elit. Sed placerat nulla a vulputate ultrices. Ut et magna ac lacus elementum tincidunt a id ante.

%     {\color{magenta}
%       Donec $u=v_1 v_2 \ldots v_m$ aenean ullamcorper odio vitae $v_i$ erat $\ell_n$  rhoncus quis
%       \[
%       e_n(u)=e_n(v_1)\ldots e_n(v_m)
%       \]
%     }
%   \end{definition}
% \end{frame}

% %%%%%%%%%%%%%%%%%%%%%%%%%%%%%%%%%%%%%%%%%%%%%%%%%%%%%%%%%%%%%%%%%%%

% \begin{frame}
%   \frametitle{Problemas abiertos}

%   Cras id accumsan risus, sed elementum elit.
%   \pause

%   \begin{itemize}
%     \item Donec eu sollicitudin lacus. Vestibulum facilisis eu tellus quis gravida. Proin faucibus tellus nec tempus maximus.
%     \pause
    
%     \item Proin at facilisis orci. Nunc at orci in ante semper elementum ullamcorper in est. Praesent maximus aliquet lorem, in tincidunt odio tempus vel.
%     \pause

%     \item Etiam vulputate nunc eget mauris vestibulum, nec viverra massa lacinia. Donec volutpat tempus nunc, vitae malesuada odio ultricies nec. 
%   \end{itemize}
% \end{frame}

\newcounter{finalframe}
\setcounter{finalframe}{\value{framenumber}}

\addtocounter{framenumber}{-1}

% \begin{frame}
% \scriptsize

% \begin{thebibliography}{1}

% \setlength{\parskip}{-0.5mm}

% \bibitem{agafonov1968normal}
% V.~N. Agafonov.
% \newblock Normal sequences and finite automata.
% \newblock {\em Soviet Mathematics Doklady}, 9:324--325, 1968.

% \bibitem{BecherCarton2017}
% Ver\'onica Becher and Olivier Carton.
% \newblock Normal numbers and computer science.
% \newblock In Val\'erie Berth\'e and Michel Rig\'o, editors, {\em Sequences,
%   Groups, and Number Theory}, Trends in Mathematics Series.
%   Birkhauser/Springer, 2017.

% \bibitem{borel1909probabilites}
% {\'E}mile Borel.
% \newblock Les probabilit{\'e}s d{\'e}nombrables et leurs applications
%   arithm{\'e}tiques.
% \newblock {\em Rendiconti del Circolo Matematico di Palermo}, 27(1):247--271,
%   1909.

% \bibitem{bugeaud2012distribution}
% Yann Bugeaud.
% \newblock {\em Distribution modulo one and Diophantine approximation}, volume
%   193.
% \newblock Cambridge University Press, 2012.

% \bibitem{Champernowne:1933}
% David Champernowne.
% \newblock The {C}onstruction of {D}ecimals {N}ormal in the {S}cale of {T}en.
% \newblock {\em The Journal of the London Mathematical Society},
%   s1-8(4):254--260, 1933.

% \bibitem{rogers1987theory}
% Jr. Hartley~Rogers.
% \newblock {\em Theory of recursive functions and effective computability}.
% \newblock MIT Press, Cambridge, MA, second edition, 1987.

% \bibitem{kamae1975normal}
% Teturo Kamae and Benjamin Weiss.
% \newblock Normal numbers and selection rules.
% \newblock {\em Israel Journal of Mathematics}, 21(2):101--110, 1975.

% \bibitem{Piatetski-Shapiro:1951}
% I.~I. Piatetski-Shapiro.
% \newblock On the law of distribution of the fractional parts of the exponential
%   function.
% \newblock {\em Izv. Akad. Nauk SSSR Ser. Mat.}, 15(1):47--52, 1951.

% \bibitem{vandehey2016uncanny}
% Joseph Vandehey.
% \newblock Uncanny subsequence selections that generate normal numbers.
% \newblock {\em ar{X}iv:1607.03531}, 2016.
% \end{thebibliography}

% \end{frame}

\end{document}
