%\begin{center}
%\large \bf \runtitulo
%\end{center}
%\vspace{1cm}
\chapter*{\runtitulo}

\noindent En este trabajo estudiamos una secuencia de números reales completamente equidistribuidos publicada por Donald Knuth en 1965. La noción de equidistribución completa se utiliza en su sentido clásico; es decir, que para una secuencia dada, todas sus subsecuencias finitas y contiguas de cualquier longitud presentan una distribución uniforme. En un artículo de 1963, Joel Franklin considera esta propiedad como un primer requerimiento de pseudoaleatoriedad en secuencias determinísticas, y prueba que la equidistribución completa implica muchas otras propiedades importantes de las secuencias aleatorias. El trabajo de Knuth se basa en secuencias de De Bruijn, las cuales tienen también una relación cercana con la noción de equidistribución y pueden ser generadas en tiempo constante amortizado por el algoritmo FKM (Fredricksen, Kessler, Maiorana, 1978). Presentamos una variante de la secuencia de Knuth mediante una construcción similar, aunque más sencilla, y damos una prueba elemental de que la secuencia generada también es completamente equidistribuida.

\bigskip

\noindent\textbf{Palabras claves:} secuencia aleatoria, equidistribución completa, secuencia de De Bruijn, algoritmo FKM, secuencia de Ford, secuencia de Knuth.
