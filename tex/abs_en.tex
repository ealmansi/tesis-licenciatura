%\begin{center}
%\large \bf \runtitle
%\end{center}
%\vspace{1cm}
\chapter*{\runtitle}

\noindent In this work, we study a construction published by Donald Knuth in 1965 yielding a completely equidistributed sequence of real numbers. Complete equidistribution is interpreted in its classic sense; namely, that finite contiguous subsequences of any length have a uniform distribution within a given sequence. Joel Franklin in a paper from 1963 suggests this as a first requirement for pseudorandomness in deterministic sequences, and proves that complete equidistribution implies many other important statistical properties shared by all random sequences. Knuth's work is based on De Bruijn sequences, which are also closely related to equidistribution and can be generated by the FKM algorithm (Fredricksen, Kessler, Maiorana, 1978) in amortized constant time. We provide a variant of Knuth's sequence via a similar, albeit simpler, construction and give an elementary proof showing that the sequence it yields is also completely equidistributed.

\bigskip

\noindent\textbf{Keywords:} Random Sequence, Complete Equidistribution, De Bruijn Sequence, FKM Algorithm, Ford Sequence, Knuth Sequence.
