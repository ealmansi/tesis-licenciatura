%\begin{center}
%\large \bf \runtitle
%\end{center}
%\vspace{1cm}
\chapter*{\runtitle}

\noindent In this work, we study a construction published by Donald Knuth in 1965 yielding a completely equidistributed sequence of real numbers which, to the best of our knowledge, is the first explicit example of a sequence exhibiting this property present in the literature. Complete equidistribution is interpreted in its classic sense; namely, that finite contiguous subsequences of any length have a uniform distribution within a given sequence. Joel Franklin in a paper from 1963 suggests this as a first requirement for pseudorandomness in deterministic sequences, and proves that complete equidistribution implies many other important properties shared by all random sequences. This concept is additionally related to normal numbers, since we say that a number is normal in a given base if its representation in such base is a completely equidistributed sequence. Knuth's work is based on De Bruijn sequences, which are also closely related to equidistribution and can be generated by the FKM algorithm in amortized constant time. We provide a variant of his sequence via a similar, albeit simpler, construction and prove that the sequence it yields is also completely equidistributed.

\bigskip

\noindent\textbf{Keywords:} Random Sequence, Complete Equidistribution, De Bruijn Sequence, FKM Algorithm, Ford Sequence, Knuth Sequence